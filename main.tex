\documentclass{article}
\usepackage[utf8]{inputenc}
\usepackage{hyperref}
\usepackage{minted}

\title{BUSA8090 - Assignment 1, Task 1}
\author{Samuel Gardiner (44952619)}
\date{17 March 2020}

\newminted{bash}{
    style = friendly
}

\begin{document}

\maketitle

\section{}

Our answer is the following \texttt{bash} script, \texttt{newest.sh}:

\begin{bashcode}

\end{bashcode}


\section{}

\section{}

\subsection{}
'Program 24' is the shell script \texttt{time-signal.sh}, printed at Wünschiers 10.11.2. The script has the URL:
\url{https://www.staff.hs-mittweida.de/~wuenschi/data/media/compbiolbook/chapter-10-shell-programming--time-signal.sh}. Since this is a publicly available URL, we can easily use \texttt{curl} to read the script from Wünschiers' webserver and write it to a directory on our Ubuntu instance with the file redirect operator, \texttt{>}.

First, we create the directory \texttt{~/bin}:
\begin{bashcode}
mkdir -p ~/bin
\end{bashcode}

We have used the \texttt{-p} argument to \texttt{mkdir} as it prevents \texttt{mkdir} from creating a directory if the directory already exists.

Now, we can write a file \texttt{~/bin/time-signal.sh} using \texttt{curl}:

\begin{minted}{bash}

\end{minted}

\end{document}
